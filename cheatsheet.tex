\documentclass[10pt,landscape]{article}
\usepackage{multicol}
\usepackage{calc}
\usepackage{ifthen}
\usepackage[landscape]{geometry}
\usepackage{hyperref}
\usepackage{listings}
\usepackage[default]{sourcecodepro}
\usepackage[T1]{fontenc}

% To make this come out properly in landscape mode, do one of the following
% 1.
%  pdflatex latexsheet.tex
%
% 2.
%  latex latexsheet.tex
%  dvips -P pdf  -t landscape latexsheet.dvi
%  ps2pdf latexsheet.ps


% If you're reading this, be prepared for confusion.  Making this was
% a learning experience for me, and it shows.  Much of the placement
% was hacked in; if you make it better, let me know...


% 2008-04
% Changed page margin code to use the geometry package. Also added code for
% conditional page margins, depending on paper size. Thanks to Uwe Ziegenhagen
% for the suggestions.

% 2006-08
% Made changes based on suggestions from Gene Cooperman. <gene at ccs.neu.edu>


% To Do:
% \listoffigures \listoftables
% \setcounter{secnumdepth}{0}


% This sets page margins to .5 inch if using letter paper, and to 1cm
% if using A4 paper. (This probably isn't strictly necessary.)
% If using another size paper, use default 1cm margins.
\ifthenelse{\lengthtest { \paperwidth = 11in}}
	{ \geometry{top=.5in,left=.5in,right=.5in,bottom=.5in} }
	{\ifthenelse{ \lengthtest{ \paperwidth = 297mm}}
		{\geometry{top=1cm,left=1cm,right=1cm,bottom=1cm} }
		{\geometry{top=1cm,left=1cm,right=1cm,bottom=1cm} }
	}

% Turn off header and footer
\pagestyle{empty}
 

% Redefine section commands to use less space
\makeatletter
\renewcommand{\section}{\@startsection{section}{1}{0mm}%
                                {-1ex plus -.5ex minus -.2ex}%
                                {0.5ex plus .2ex}%x
                                {\normalfont\large\bfseries}}
\renewcommand{\subsection}{\@startsection{subsection}{2}{0mm}%
                                {-1explus -.5ex minus -.2ex}%
                                {0.5ex plus .2ex}%
                                {\normalfont\normalsize\bfseries}}
\renewcommand{\subsubsection}{\@startsection{subsubsection}{3}{0mm}%
                                {-1ex plus -.5ex minus -.2ex}%
                                {1ex plus .2ex}%
                                {\normalfont\small\bfseries}}
\makeatother

% Define BibTeX command
\def\BibTeX{{\rm B\kern-.05em{\sc i\kern-.025em b}\kern-.08em
    T\kern-.1667em\lower.7ex\hbox{E}\kern-.125emX}}

% Don't print section numbers
\setcounter{secnumdepth}{0}


\setlength{\parindent}{0pt}
\setlength{\parskip}{0pt plus 0.5ex}


% -----------------------------------------------------------------------

\begin{document}

\raggedright
\footnotesize
\begin{multicols}{2}


% multicol parameters
% These lengths are set only within the two main columns
%\setlength{\columnseprule}{0.25pt}
\setlength{\premulticols}{1pt}
\setlength{\postmulticols}{1pt}
\setlength{\multicolsep}{1pt}
\setlength{\columnsep}{2pt}

\begin{center}
     \Large{\textbf{DTrace Cheat Sheet}} \\
\small{\verb|dtrace -n 'probe /predicate/ { action; }'|}
\end{center}

\section{FINDING PROBES}
\begin{tabular}{@{}ll@{}}
  \verb+dtrace -l | grep foo+ & keyword search\\
  \verb|dtrace -n 'fbt:::entry { @[probefunc] = count(); }'| & frequency count \\
\end{tabular}


\section{PROBE ARGUMENTS}
\begin{tabular}{@{}ll@{}}
\verb|syscall:::| & \verb|man syscallname|\\
\verb|fbt:::| & Kernel functions
\end{tabular}

\section{PROBES}
\begin{tabular}{@{}ll@{}}
\verb|BEGIN| & D program start\\
\verb|END| & D program end\\
\verb|tick-1sec| & run once per sec, one CPU only\\
\verb|syscall::read*:entry| & process reads\\
\verb|syscall::write*:entry| & process writes\\
\verb|syscall::open*:entry| & file open\\
\verb|proc:::exec-success| & process create\\
\verb|io:::start,io:::done| & storage I/O\\
\verb|lockstat:::adaptive-block| & kernel blocking mutex\\
\verb|sched:::off-cpu| & thread leaves a CPU\\
\verb|fbt:::entry| & all kernel functions\\
\verb|profile-123 | & sample at 123Hz\\
\end{tabular}

\section{D VARIABLES}
\begin{tabular}{@{}ll@{}}
\verb|execname| & on-CPU process name \\
\verb|pid, tid| & on-CPU PID, Thread ID\\
\verb|cpu| & CPU ID\\
\verb|timestamp| & time, nanoseconds\\
\verb|vtimestamp| & time thread was on-CPU (ns)\\
\verb|arg0...N| & probe args (uint64)\\
\verb|args[0]...[N]| & probe args (type'd)\\
\verb|curthread| & pointer to current thread \\
\verb|probemod| & module name\\
\verb|probefunc| & function name\\
\verb|probename| & probe name\\
\verb|self->foo| & thread local variable\\
\verb|this->foo| & clause local variable\\
\verb|$1...$N| & arguments as integers\\
\verb|$$1...$$N| & arguments as strings\\
\verb|$target| & -p PID, -c command\\
\verb|curpsinfo| & process information\\
\end{tabular}

\section{ACTIONS}
\begin{tabular}{@{}ll@{}}
\verb|@agg[key1, key2] = count()| & frequency count\\
  \verb|@agg[key1, key2] = sum(var)| & sum variable\\
  \verb|@agg[key1, key2] = quantize(var)| & power of 2 quanization\\
  \verb|printf("format", var0...varN)| & print vars\\
  \verb|stack(num), ustack(num)| & print num lines of kernel or user
                                   stack\\
  \verb|func(pc), ufunc(pc)| & return kernel/user function name from a
                               PC\\
  \verb|clear(@)| & clear an aggregation\\
  \verb|trunc(@, 5)| & truncate aggreation to top 5 entries\\
  \verb|stringof(ptr)| & string from kernel address\\
  \verb|copyinstr(ptr)| & string from user address\\
  \verb|exit(0);| & exit dtrace\\
\end{tabular}

\section{SWITCHES}
\begin{tabular}{@{}ll@{}}
\verb|-n| & trace the given probe point\\
\verb|-l| & list probes instead of tracing\\
\verb|-q| & quiet; don't print default output\\
  \verb|-s <file>| & invoke a D script file\\
\verb|-w| & allow destructive actions\\
\verb|-p PID| & allow \verb|pid:::| provider to trace this pid; it's also \verb|$target|\\
\verb|-c 'command'| & have dtrace invoke a command\\
\verb|-o file| & send trace output to a file\\
\verb|-x options| & set various DTrace options (switchrate, bufsize...)\\
\end{tabular}

\section{PRAGMAS}
\begin{tabular}{@{}ll@{}}
\verb|#pragma D option quiet| & same as \verb|-q|, quiet output\\
\verb|#pragma D option destructive| & same as \verb|-w|, allow
                                      destructive actions\\
  \verb|#pragma D option switchrate=10hz| & print at 10Hz instead of
                                            1Hz\\
  \verb|#pragma D option bufsize=16m| & set per-CPU buffer size\\
  \verb|#pragma D option defaultargs| & \verb|$1| is 0, \verb|$$1| is ""\\
\end{tabular}

\section{ONE-LINERS}
\begin{lstlisting}
dtrace -n 'proc:::exec-success{ trace(curpsinfo->pr_psargs); }'
dtrace -n 'syscall:::entry { @num[execname] = count(); }'
dtrace -n 'syscall::open*:entry { printf("%s %s", execname, copyinstr(arg0)); }' dtrace -n 'io:::start { @size = quantize(args[0]->b_bcount); }'
dtrace -n 'fbt:::entry { @calls[probemod] = count(); }'
dtrace -n 'sysinfo:::xcalls { @num[execname] = count(); }'
dtrace -n 'profile-1001 { @[stack()] = count() }'
dtrace -n 'profile-101 /pid == $target/ { @[ustack()] = count() }' -p PID dtrace -n 'syscall:::entry { @num[probefunc] = count(); }'
dtrace -n 'syscall::read*:entry { @[fds[arg0].fi_pathname] = count(); }'
dtrace -n 'vminfo:::as_fault { @mem[execname] = sum(arg0); }'
dtrace -n 'sched:::off-cpu /pid == $target/ { @[stack()] = count(); }' -p PID dtrace -n 'pid$target:libfoo::entry { @[probefunc] = count(); }' -p PID
\end{lstlisting}
\end{multicols}
\end{document}

%%% Local Variables:
%%% mode: latex
%%% TeX-master: t
%%% End:
